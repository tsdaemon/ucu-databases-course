\documentclass[12pt]{article}
\usepackage[utf8]{inputenc}
\usepackage{hyperref}
\usepackage{indentfirst}

\hypersetup{ colorlinks, citecolor=green, filecolor=black, linkcolor=blue, urlcolor=blue }

\newcommand{\code}[1]{\texttt{#1}}

\begin{document}

\title{Practical assignment 3: Aggregation}
\author{Anatolii Stehnii\\Intro to databases}
\maketitle

\section*{Description}

The purpose of this assignment is to create an aggregated representation of stored information. Additionally, you will study code decomposition with help of Dependency Injection and pattern IRepository.

\section*{Conventions}

This assignment will be tested by automated functional tests, therefore all your manual actions in a database (table creation, user creation etc) should be reflected in a corresponding \code{.sql} files. I strictly advise you to use environment setup (DotNetCore2.0+MySQL) described in assignment 0 to ensure compatibility of your results with testing and production environments.

All file names are specified relative to your \code{ucubot} repository location.

Note that for C\# an accepted convention is to use \code{CamelCase} identifiers, while in MySQL names are usually \code{snake\_case}. Therefore a table \code{lesson\_signal} corresponds to a class and an UML diagram \code{LessonSignal}.

\section*{Recommendations}

Each practical assignment could contain some information not covered in lectures. But this should not stop you from its fulfillment. I encourage you to use Google to fill gaps in your practical knowledge. However, to reduce ambiguity between different information sources, I have added a list of recommended links with hints for each unknown point at the end of the assignment. It is no shame to use it, but first, try to find information by yourself.

If you have any issues with this assignment, please reach me at Slack @Anatolii Stehnii or send an email to \href{mailto:stehnii@ucu.edu.ua}{stehnii@ucu.edu.ua}. Use course Slack channel to ask questions and help your colleagues. If you have any new ideas or improvements, please send pull-request to repository \href{https://github.com/tsdaemon/ucu-databases-course}{ucu-databases-course}. Your colleagues will be grateful if you make this instruction better.

\section*{Instructions}

\subsection*{Repositories}

As described in topic Application 3, mixing HTTP and SQL responsibilities in controllers are breaking principle of Single Responsibility. Therefore we need to decompose our code into separate modules. All code related to an input arguments processing and HTTP codes handling should stay inside of controllers. All code related to a database should be carried out into Repositories.

\begin{enumerate}
\item Decompose \code{StudentEnpointController}:
    \begin{enumerate}
    \item Create a class for a student repository and put in it all database code from the controller. Hint: there is no directory for database code in your project yet, but you can create one!
    \item Create an interface \code{IStudentRepository} which defines an abstract contract for the previously created repository.
    \item Reference your repository as an implementation of \code{IStudentRepository} in \code{Startup.cs}.
    \item Register dependency \code{IStudentRepository} in \code{StudentEnpointController} and replace all database related code with its functions.
    \end{enumerate}
\item Decompose \code{LessonSignalEndpointController} in the same way.
\end{enumerate}

Think: what line is often repeated in your database code? Can it be a separate responsibility? How can you write a repository, which is independent of an actual DBMS (MySQL, Postgres etc)?

\subsection*{Aggregated report}

The previous section described only a refactoring of previously created code. Now we will add new functions to the application. We will create a view to show a report about a number of different signals for each student. Use a class definition below as a reference:

\begin{verbatim}
class StudentSignal {
    FirstName: string,
    LastName: string,
    SignalType: string, allowed values - [Simple, Normal, Hard],
    Count: int
}
\end{verbatim}

The class \code{StudentSignal} should correctly override Equals() method.

\begin{enumerate}
\item Connect to MySQL. Write a select query which aggregates a number of lesson signals by type by student. Transform numerical signal types into meaningful type names (\code{Simple, Normal, Hard}). Save this select as a view \code{student\_signals}. Save your code into \code{Scripts/student-signals.sql}.
\item Create a class \code{StudentSignal} to represent records of this view.
\item Create a new repository with a query to this view.
\item Create a new controller \code{StudentSignalEndpointController} with one \code{HttpGet} action, which shows all records in \code{student\_signals} view.
\end{enumerate}

\section*{Submit}

Please DO NOT SHARE your code with the colleagues or in the Slack channel!

To prevent code stealing from PRs this assignment should be submitted in the following way:
\begin{enumerate}
\item Make you repository private.
\item Add me (\code{tsdaemon}) as a repository collaborator.
\item Submit a link to your working branch in the CMS.
\end{enumerate}

\clearpage

\section*{Hints}
How to \dots
\begin{itemize}
\item \dots \href{https://docs.microsoft.com/en-us/aspnet/core/fundamentals/dependency-injection?view=aspnetcore-2.1}{use dependency injection in C\#}.
\item \dots register dependency for injection - check lecture Application 3.
\item \dots \href{https://www.codeproject.com/Articles/10197/The-Interface-Construct-in-C}{create a module interface}.
\item \dots \href{https://dev.mysql.com/doc/refman/8.0/en/create-view.html}{create a view}.

\end{itemize}


\end{document}
